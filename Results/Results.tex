\newpage
\section{Results and Discussion}

This section summarizes the results obtained from the Halo Assay, BS Assay, and Jar Assay, focusing on bacterial interactions, fungal inhibition, and plant growth effects.

\par
The first set of data analyzed focused on the effects of different bacterial strains on the root weight and root length of barley plants. These results are visualized in \autoref{fig:root_weight_boxplot} and \autoref{fig:root_length_boxplot} respectively.
Both figures are sorted by the mean of the root weight to enable better comparison. 


\begin{figure}[H]
    \makebox[\textwidth]{ % Keeps the figure centered within the text width
    \includesvg[width=1.1\textwidth, keepaspectratio]{root_weight_boxplot}
    }
    \caption{Distribution of root weight (in mg) for barley plants treated with different bacterial strains, sorted by the median root weight.}
    \label{fig:root_weight_boxplot}
\end{figure}% Root Length Boxplot

\begin{figure}[htbp]
    \centering
    \setlength{\abovecaptionskip}{-30pt} % Adjust only for th
    \includesvg[pretex=\tiny, width=1.1\textwidth]{root_length_boxplot}
    \caption{Distribution of root lengths (in cm) for barley plants treated with d ifferent bacterial strains, sorted by the median root weight}
    \label{fig:root_length_boxplot}
\end{figure}


The interactions between the bacterial strains, that were evaluated using the Halo Assay are visualized in \autoref{fig:heatmap} in the form of a heatmap. 
This heatmap represents the antagonistic effects observed between bacterial strains, where the size and color of the dots correspond to the size and strength of the inhibition halos, respectively. Small gray dots indicate the absence of halo, despite of bacterial growth, while the absence of a dot indicates no bacterial growth in that interaction. 

Some bacterial strains stood out by producing notably larger inhibition halos, indicating strong antagonistic effects. Most notably, strains 1208, 1847 consistently generated the largest halos, suggesting an ability to inhibit the growth of other strains. These strains displayed both substantial halo size and high halo strength, as represented by darker colors in \autoref{fig:heatmap}.

Certain bacterial strains were unable to establish growth when interacting with any of the other strains, as indicated by the absence of dots in \autoref{fig:heatmap}. This suggests that these strains were either highly sensitive to antagonistic effects from other strains or lacked the competitive ability to survive in such interactions under the experiment conditions. 
For example, strains 1114 and Y showed growth in only a few cases without producing measurable halos, suggesting a neutral or non-antagonistic role within the community.
% Heatmap
\begin{figure}[H]
    \centering
    \setlength{\abovecaptionskip}{-30pt} % Adjust only for this figure
    \makebox[\textwidth]{ % Keeps the figure centered within the text width
        \includesvg[width=1.2\textwidth, keepaspectratio]{halo_assay_heatmap}
    }
    \caption{Bacterial interaction heatmap visualizing the interactions between bacterial strains, with halo size and strength indicating antagonistic effects. Grey dots represent no observed halo while the absence of a dot indicated no bacterial growth.}
    \label{fig:heatmap}
\end{figure}

\begin{figure}[H]
    \centering
    \footnotesize
    \setlength{\abovecaptionskip}{-30pt} % Adjust only for this figure
    \makebox[\textwidth]{ % Keeps the figure centered within the text width
        \includesvg[width=1.2\textwidth, keepaspectratio]{inhibition_barplot}
    }
    \caption{Total inhibition by bacterial strain. The bar plot shows the total instances of inhibition observed for each bacterial strain. Colors indicate the bacterial class. Strains are ordered by their total inhibition count.}
    \label{fig:inhibition}
\end{figure}
\vspace{0.5cm} % Add space between the figure and the detailed explanation

\begin{figure}[H]
    \centering
    \footnotesize
    \setlength{\abovecaptionskip}{-30pt} % Adjust only for this figure
    \makebox[\textwidth]{ % Keeps the figure centered within the text width
        \includesvg[width=1.2\textwidth, keepaspectratio]{retention_factor_barplot}
    }
    \caption{Mean retention factor by bacterial strain. The retention factor is a relative value indicating how much the bacterial strain inhibited or retained the fungal pathogen. Colors represent the bacterial class. Strains are ordered by their mean retention factor.}
    \label{fig:retention}
\end{figure}
\vspace{0.5cm} % Add space between the figure and the detailed explanation


\noindent
The retention factor is calculated using the formula:
\[
\text{Inhibition (\%)} = \frac{\text{Control Distance} - \text{Bacteria Distance}}{\text{Control Distance}} \times 100
\]
where the \textit{Control Distance} refers to the growth measurement in the absence of bacterial interference, and the \textit{Bacteria Distance} refers to the growth measurement in the presence of bacterial strains. Higher retention factor values indicate greater inhibition by the bacterial strain. The bar plots in \autoref{fig:inhibition} and \autoref{fig:retention} visualize the effect of the pathogens on fungal growth for each bacterial strain, highlighting differences in inhibitory effects across strains.


Strain 1847 that was classified via the Type Strain Genome Server (TYGS), as can be seen in \autoref{tab:strains}, as \textit{Pseudomonas chlororaphis} conistently showed strong result across the three assay. This indicates a significant antagonistic avtivity towards other strains, effective inhibition of \ac{Bipo} growth and a notable influnce on root growth in the Jar Assay.

This strain was already subject of a study published in 2021 by Bertani et. al., showing several effect both on plant's and fungi. 
In the study, the \textit{Pseudomonas chlororaphis} strain effectively inhibited fungal pathogens like \textit{Fusarium graminearum} through antimicrobial production and colonized plant roots, enriching beneficial microbes in the rhizosphere. While it influenced plant stress pathways, it showed no direct growth-promotion effects under the tested conditions. \cite{bertani2021Isolation}

This correlated with our finings, as the strain reduced plant root growth and length while demonstrating  strong antagonistic effects on other bacterial strains and weak antagonistic effects on the fungal pathogen.