\section{Introduction}

\subsection{The Role of the Plant Microbiome}
In their natural environment, plants are in constant interaction with a diverse community of microorganisms, known as the plant microbiota. These include bacteria, fungi, archaea, viruses, and even protists, which colonize various plant surfaces such as roots, leaves, and stems, as well as the internal tissues.
The plant microbiome has been recognized as a crucial factor influencing plant health and productivity for more than a century \cite{berg2016Plant}.
Many functions of the plant microbiome are essential for plant health and development. 

From the beginning of a plant's life cycle, microorganisms play a key role. During germination, certain plants rely heavily on microorganisms. Mosses, some of the oldest land plants, depend on microbial interactions to kickstart their growth \cite{hornschuh2006Mossassociated}. 
Similarly, orchids require the assistance of fungi to germinate successfully \cite{jacquemyn2015Mycorrhizal}. 

Throughout a plant’s life, its microbiome remains integral to its fitness, influencing both pathogen suppression and nutrient acquisition, particularly in the rhizosphere.
The rhizosphere microbial community acts as a protective shield agaisnt soil-born pathogens, directly inhibiting harmful microbes while priming the plant's immune system to enhance resistance \cite{berg2016Plant}. At the same time, the microbial communities facilitate the solubilization and uptake of essential nutrients such as nitrogen, phosphorus, and iron, thereby directly contributing to plant growth, crop yields. Due to the coevolution and interdependence between plants and their associated microbes, the concept of the holobiont emerges, representing them as a unified ecological unit \cite{spooren2024PlantDriven}.

During germination, seed-associated microbes are most critical for the inital stages of microbiome development. As the plant matures, however, soil microbes usually become the major constituents of the plant microbiome. 
These plant-microbe relationsships extend beyond passive colonization, as plants actively shape their microbiome. 

One of the most influential processes is the release of root exudates, a mixture of primary and secondary metabolites. Primary metabolites such as sugars, organic acids, and amino acids serve as nutrient sources, selectively attracting fast-growing and metabolically versatile microbes to the rhizosphere \cite{spooren2024PlantDriven}.

Secondary metabolites play a regulatory role by either promoting the growth of beneficial microbes or suppressing harmful ones. For example, plants release antimicrobial compounds that deter pathogens, signaling molecules that influence microbial behavior, or toxic substances that create an unfavorable environment for specific microbes. Together, these compounds help shape the composition of the plant microbiome, fostering a community that supports plant health and growth. \cite{spooren2024PlantDriven}.

\subsection{The SynCom approach}
Understanding the complex web of plant-microbe interactions requires tools to study simplified microbial communities and their roles in the microbiome. A \ac{SynCom} consist of defined microbial strains assembled to model natural microbiomes. 
Using this approach, it becomes possible to study plant-microbe and microbe-microbe interactions under controlled conditions, enabling the identification of beneficial microbes and mechanisms that can enhance plant health and resilience. By combining different microbial taxa, a \ac{SynCom} enables the study of both synergistic and antagonistic relationships within microbial communities.
The \ac{SynCom} used in this study was created by isolating bacterial strains from the roots of barley (Hordeum vulgare), a host with a well-defined microbiota. Plants were grown in natural soil collected from the \textit{"Kölner Loch"} \footnotemark \footnotetext{Nick Dunken} near the Max Planck Institute, ensuring that the microbial community was representative of a realistic environment. Bacterial strains were isolated directly from barley roots, as microorganisms that interact most closely with plants are typically concentrated in this region.
The \ac{SynCom} was designed to include members of the core microbiota consistently associated with barley across diverse environments. The strains were selected based on their ecological relevance and potential functional roles in the microbiome, such as nutrient cycling, pathogen suppression, and plant growth promotion. The strains, which represent representing a potential \ac{SynCom} for investigating interactions within the barley root microbiota, are listed in Table~\ref{tab:strains}.

\subsection{Research Gap and Objectives}
Synthetic communities have been widely utilized to investigate plant-microbe interactions, including in barley. Prior studies have demonstrated the potential of multiparte SynComs to protect barley against pathogens such as Bipolaris sorokiniana, enhance plant growth, and improve resilience to abiotic stresses like drought. Inter-kingdom SynComs that combine bacterial and fungal strains have further highlighted the synergistic potential of these communities in promoting plant health. \cite{mahdi2022Fungal}. These findings underscore the utility of SynComs in advancing our understanding of complex plant-microbe systems.

However, in the context of this project, initial experiments with the selected SynCom failed to yield a positive effect on barley plants. In fact, certain observations suggested potential negative impacts, raising questions about the dynamics within this specific SynCom. The precise interactions among the selected bacterial strains, as well as individual effect on the plant and the fungal pathogen, remain poorly understood.

Addressing these gaps could provide critical insights into the mechanisms driving these interactions, possibly enabling the development of an optimized SynCom that fosters plant growth and resilience. By identifying factors such as antagonistic interactions, pathogen-promoting effects, or direct negative impacts on the plant.

This study focuses on a small but critical aspect of these challenges by examining the individual effects of each strain. By identifying potential antagonistic interactions, pathogen-promoting effects, or negative impacts on the plant, this project aims to contribute to a better understanding of the SynCom dynamics. These findings may help further efforts to design a \ac{SynCom} with a more beneficial effect on barley growth.

\begin{table}[!ht]
    \caption{Bacterial Strains in the SynCom}
    \label{tab:strains}
    \centering
    \makebox[\textwidth][c]{ % Center the table within the page
        \resizebox{1.1\textwidth}{!}{
    \begin{tabularx}{\textwidth}{|>{\centering\arraybackslash}p{1.5cm}|X|>{\centering\arraybackslash}p{1cm}|>{\centering\arraybackslash}p{2cm}|>{\centering\arraybackslash}p{2cm}|>{\centering\arraybackslash}X|}
        \hline
        \textbf{Strain number} & \textbf{Genus} & \textbf{Gram} & \textbf{Genus enriched in GP} & \textbf{Genus \linebreak beneficial} & \textbf{TYGS} \\ \hline
        428 & Acidovorax & - & No & Yes & Not found \\ \hline
        459 & Ensifer & - & No & Yes & Ensifer canadensis \\ \hline
        638 & Cellulomonas & + & No & Yes & Not found \\ \hline
        978 & Nocardioides & + & No & Yes & Not found \\ \hline
        997 & Pseudomonas & - & Yes & Yes & Pseudomonas cedrina \\ \hline
        1080 & Devosia & - & No & Yes & Not found \\ \hline
        1101 & Shinella & - & No & Yes & Not found \\ \hline
        1114 & Galbitalea & + & No & unidentified & Not found \\ \hline
        1208 & F\_Micrococcaceae & + & No & Yes & Not found \\ \hline
        1234 & Neorhizobium & - & No & Yes & Not found \\ \hline
        1334 & Arthrobacter & + & No & Yes & Not found \\ \hline
        1338 & Chitinophaga & - & Yes & Yes & Not found \\ \hline
        1350 & Microbacterium & + & No & Yes & Not found \\ \hline
        1362 & Lysobacter & - & Yes & Yes & Not found \\ \hline
        1475 & Hydrogenophaga & - & No & Yes & Not found \\ \hline
        1533 & Lysobacter & - & Yes & Yes & Not found \\ \hline
        1692 & Sphingomonas & - & Yes & Yes & Not found \\ \hline
        1703 & Devosia & - & No & Yes & Not found \\ \hline
        1725 & Sphingomonas & - & Yes & Yes & Sphingomonas \linebreak lacusdianchii \\ \hline
        1790 & Sphingomonas & - & Yes & Yes & Not found \\ \hline
        1847 & Pseudomonas & - & Yes & Yes & Pseudomonas \linebreak chlororaphis \\ \hline
        2751 & Bacillus & + & No & Yes & Priestia megaterium \\ \hline
        2998 & Paenibacillus & - & No & Yes & Paenibacillus \linebreak humicus \\ \hline
        3044 & Pseudomonas & - & Yes & Yes & Not found \\ \hline
    \end{tabularx}
    }
}
    \par
    \vspace*{0.2cm}
    \raggedright
    \textbf{Description:} Bacterial strains used in the \ac{SynCom} for studying barley root microbiota interactions. Columns include (left to right): strain ID, genus, gram type, weather the strain was enriched in Golden Promise compared to Hit4, plant-beneficial status of the genus, and classification via the Type Strain Genome Server (TYGS). The strain numbers are arbitrary and were assigned in previous studies.
\end{table}