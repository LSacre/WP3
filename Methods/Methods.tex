\section{Methods}

\subsection{Strain Selection}



\subsection{Overview of Experimental Design}
To investigate the interactions within the bacterial \ac{SynCom} and their effects on barley and \textit{Bipolaris sorokiniana} (hereafter referred to as Bipo), three distinct experiments were conducted:
\begin{enumerate}
    \item \textbf{Halo Assay}: Testing interactions within the bacterial \ac{SynCom}.
    \item \textbf{Bipo Assay}: Evaluating bacterial interactions with the fungal pathogen \linebreak \ac{Bipo}.
    \item \textbf{In Planta Jar Assay}: Assessing the effects of individual bacterial strains on barley plants.
\end{enumerate}
Each experiment focused on a specific aspect of plant-microbe or microbe-microbe interactions.


\subsection{Halo Assay: Testing Interactions Within Strains}
The halo assay was adapted from the method described by \citet{getzke2023Cofunctioning}. to assess antagonistic interactions between bacterial strains in the \ac{SynCom}.

Liquid cultures of each strain were grown in \ac{TSB} and standardized to an \ac{OD} of 1 at 600 nm. For each plate, 600 µL of bacterial culture at OD600 = 1 was added to 50 g of molten \ac{TSA} before solidifying to create base plates for the assay.

Drops of 10 µL of liquid bacterial cultures, diluted to OD600 = 0.1, were applied to the solidified plates, ensuring that each strain interacted with every other strain exactly once. Plates were incubated and inhibition halos around the droplets were measured to quantify antagonistic interactions. This approach,  has been shown to effectively study microbial community dynamics \cite{getzke2023Cofunctioning}.

\pagebreak

\subsection{Bipo Assay: Testing Interactions with Bipo.}
The goal of the Bipo Assay was to determine whether bacterial strains in the \ac{SynCom} inhibited the growth of the fungal pathogen Bipolaris sorokiniana. For the assay, Bipolaris sorokiniana was stamped onto fresh CMBS plates to establish fungal growth. Drops of liquid bacterial cultures, prepared as described in the Halo Assay, were applied to the same plates in proximity to the fungal colony.

Plates were incubated at 27° C for 6 days, and fungal growth inhibition was assessed by measuring inhibition zones and weather the fungus could grow over the bacterial colony. Negative controls should have included plates with Bipolaris sorokiniana alone, without bacterial addition.
% Add more specific details here.

\subsection{In Planta Jar Assay: Testing Effects on Barley Plants}
The Jar Assay was conducted to evaluate the effects of individual bacterial strains in the \ac{SynCom} on barley (\textit{Hordeum vulgare}) growth.
Barley seeds were germinated and transferred to jars containing a standard sterile growth medium (e.g., soil or sand). A single inoculation of liquid bacterial culture was applied to the seedlings, with each strain tested individually. Mock-treated plants, inoculated with sterile water instead of bacterial cultures, served as negative controls.

Plants were grown under standard greenhouse conditions appropriate for barley cultivation.
Measurements were taken after an appropriate growth period, sufficient to assess the effects of the bacterial treatments on plant development.

The following plant growth parameters were measured:
\begin{itemize}
    \item Root weight
    \item Root length
    \item Shoot length
\end{itemize}

This setup allowed for the systematic assessment of the effects of each bacterial strain on plant growth and health.
% Add more specific details here.